\documentclass[xcolor={table,usenames,dvipsnames},aspectratio=169]{beamer}

\usetheme[sectionpage=progressbar,subsectionpage=progressbar,block=fill]{metropolis}

\usepackage{amsthm}
\usepackage{amssymb}
\usepackage{amsmath}
\usepackage{xspace}
\usepackage{mathtools}
\usepackage{bibentry}
\usepackage[scale=1.5]{ccicons}
\usepackage{mathpartir}
\usepackage{hyperref}
\usepackage{amsmath,amssymb} % For \flalign, \top, ...
\usepackage{csquotes}
\usepackage{listings}
\usepackage{xargs}
\usepackage{transparent}
\usepackage{comment}
\usepackage{xfrac}
\usepackage[most]{tcolorbox}
\usepackage{mathpazo}

\definecolor{darkgray}{rgb}{0.31,0.31,0.33}

\definecolor[named]{keyword}{HTML}{e0004a}
\definecolor[named]{constructor}{HTML}{009304}
\definecolor[named]{darkviolet}{HTML}{1b0160}
\definecolor[named]{darkgreen}{HTML}{037f32}

\newcommand{\DARKGREEN}[1]{{\color{green!30!black}#1}}
\newcommand{\BLACK}[1]{{\color{black}#1}}
\newcommand{\RED}[1]{{\alert{#1}}}
\newcommand{\DARKRED}[1]{{\color{red!50!black}#1}}
\newcommand{\DARKBLUE}[1]{{\color{blue!50!black}#1}}
\newcommand{\BLUE}[1]{{\color{blue}#1}}
\newcommand{\MAGENTA}[1]{{\color{magenta}#1}}
\newcommand{\YELLOW}[1]{{\color{yellow}#1}}
\newcommand{\WHITE}[1]{{\color{white}#1}}
\definecolor{darkgray}{rgb}{0.31,0.31,0.33}
\newcommand{\GRAY}[1]{{\color{darkgray}#1}}

\newenvironment{mybox}[1]{\begin{block}{#1}}{\end{block}}
\newenvironment{tctsays}{\begin{mybox}{\large \tct\ says}}{\end{mybox}}
\newenvironment{question}[1]{\begin{block}{\MAGENTA{Question~{#1}}}}{\end{block}}
\newenvironment{answer}{\begin{block}{\MAGENTA{Answer}}}{\end{block}}

\lstset{%
  basicstyle=\scriptsize\ttfamily,
  numbers=left,%
  numberstyle=\tiny\color{black},%
  numbersep=10pt,%
  backgroundcolor=\color{bg},%
  frame=none,%
  %rulecolor=\color{lipicsYellow},%
  framerule=0pt,%
}

% If we want article-style math even in the presentation.
%\usefonttheme[onlymath]{serif}

\newcommand{\red}[1]{{\color{red} #1}}
\newcommand{\blue}[1]{{\color{blue} #1}}
\newcommand{\green}[1]{{\color{darkgreen} #1}}

\newcommand*{\atlas}{\ensuremath{\mathsf{ATLAS}}}

% Parameters of \seq with their default value.
% 1  start index    1
% 2  variable name
% 3  end index      n
% 4  operator       ,
% 5  dots           \dots
\newcommandx*\seq[5][1=1, 3=n, 4={,}, 5=\dots]{#2_{#1}#4#5#4#2_{#3}}
% Parameters of \seqx are the same except that the second parameter is
% an expression where \i will be replaced by the start and end index
% respectively. This allows to express sequences of more complex terms.
\newcommandx*\seqx[5][1=1, 3=n, 4={,}, 5=\dots]{{\renewcommand{\i}{#1}#2}#4#5#4{\renewcommand{\i}{#3}#2}}

\newcommand*{\yellowemph}[1]{%
\tikz[baseline=(X.base)] \node[rectangle, fill=yellow, fill opacity=0.15, text opacity=1, inner sep=0.4mm, rounded corners] (X) {#1};%
}

\title{\texorpdfstring{Software Engineering Practices for Reproducible Science}{Software Engineering Practices for Reproducible Science}}
%\subtitle{}
\author{\texorpdfstring{\textbf{Lorenz Leutgeb} \inst{1,2}}{Lorenz Leutgeb <lorenz@mpi-inf.mpg.de>}}
\institute{\inst{1} Max Planck Institute for Informatics, Saarland Informatics Campus, Germany \and \inst{2} Saarbrücken Graduate School of Computer Science, Saarland Informatics Campus, Germany}
%\institute{\vspace{10mm} \begin{center}\ccbyncsa \\[1mm] \tiny{This work is licensed under \href{http://creativecommons.org/licenses/by-nc-sa/4.0/}{CC BY-NC-SA 4.0}.}\end{center}}
\date{2022-05}

%\titlegraphic{%
%\begin{minipage}[t][3.5cm][t]{\textwidth}
%\begin{center}
%{\hspace{2.5cm}}
%\end{center}
%\end{minipage}
%}

%\logo{\includegraphics[height=1cm]{uibk_logo_4c_cmyk.pdf}}

\nobibliography*

\begin{document}
 
\begin{frame}[plain]
\maketitle
\end{frame}

%\begin{frame}{Agenda}
%\tableofcontents
%\end{frame}

\begin{frame}{Recap}
Kai Horstmann
\begin{itemize}
\item{Acknowledged that reproducibility requires data, but also code.}
\end{itemize}

Lee Cronin
\begin{itemize}
\item{Develop digital tools to aid reproducibility in chemistry.}
\item{Programming language for chemistry (implemented in Python).}
\item{Acknowledged that Python scripts are not reliable.}
\end{itemize}

Shreshth Saxena
\begin{itemize}
\item{Highlighted the benefits of leveraging free software.}
\item{Some journals now accept Jupyter notebooks for publication.}
\end{itemize}

Sandor Brockhauser
\begin{itemize}
\item Provide context for data, add an application.
\end{itemize}
\end{frame}

\begin{frame}[allowframebreaks]{Reproducibility in Software Engineering}

\begin{block}{Bug}
A bug is an error, flaw in computer software that causes it to produce
an incorrect or unexpected result or to behave in unintended ways.
\end{block}

\begin{block}{Reproducible Bug}
A bug is \emph{reproducible} if the incorrect result or unexpected behaviour
can be consistently observed and deliberately provoked
by the software developer.
\end{block}

%A bug that is \emph{not} reproducible is hard to analyze and consequently
%hard to fix, repair.

\begin{block}{Heisenbug}
A bug is a heisenbug if it cannot be consistently observed
when one attempts to analyze it.
\end{block}

\framebreak

\begin{block}{Build}
The description of the process translating software in its source
form (e.g.\ code and specification of dependencies) to its target form
(e.g.\ executable binary, PDF, collection of files to be served by a webserver).
Building means executing such a description.
\end{block}

\begin{block}{Reproducible Build}
A build is \emph{reproducible}, if its execution is deterministic,
in the sense that it always produces the same result and
it does not matter on which machine, at which time,
at which room temperature, the build is executed.
\end{block}

Helps with reproducibility of bugs, verifying builds of other parties, and caching.
\end{frame}

\begin{frame}{Bird's-Eye View}
Overarching trend: \enquote{software is eating the world} also applies to science.

Computer science is at an extreme position.
Sometimes results \emph{are} software.

Scientific artifacts inherit all problems, and their solutions, from software engineering.
\end{frame}


\begin{frame}{We're not there yet\dots}
  \begin{enumerate}
    \item<1->{Sharing \textbf<1>{data} alone is not enough!}
    \item<2->{Sharing \textbf<2>{data and code} is still not enough!}
    \item<3->{Sharing \textbf<3>{data and code and dependencies} is still not enough!}
    \item<4->{Sharing \textbf<4>{data and code and depedencies and a reproducible build} is required!}
  \end{enumerate}
\end{frame}

\begin{frame}[allowframebreaks]{Case Study: XDL}

\begin{itemize}
\item{XDL is an XML-like language to describe chemical reactions.}
%The repository at
\item{\url{https://gitlab.com/croningroup/chemputer/xdl}
contains code to parse/validate XDL, translate from a textual to a graph-based
representation, and an executor that interfaces with other components of the
chemputer.
}
\item{Python dependency management is complicated
(\texttt{setup.py}, \texttt{requirements.txt}, pip, Pipenv, Poetry).}
\end{itemize}

\framebreak

%But wait, there's more \dots
Quoting from \texttt{.gitlab-ci.yml}
\lstinputlisting[firstnumber=13,firstline=13,lastline=25]{xdl-gitlab-ci.yml}

Quoting from \texttt{setup.py}
\lstinputlisting[firstnumber=14,firstline=14,lastline=25]{xdl-setup.py}
\end{frame}

\begin{frame}{Pinning/Locking Dependencies}

\begin{center}
\begin{tabular}{lll}
Language & Package Manager & Filename \\
\hline
Java & \href{https://docs.gradle.org/7.4.2/userguide/dependency\_locking.html}{Gradle} & \texttt{gradle.lockfile} \\
%gradle dependencies --write-locks
Python & pip (and pip-tools) & \texttt{requirements.txt} \\
%pip-compile --generate-hashes --output-file=requirements.txt requirements.in
Julia & Pkg & \texttt{Manifest.toml} \\
%https://pkgdocs.julialang.org/v1.7.1/managing-packages/#Pinning-a-package
JavaScript & npm & \texttt{package-lock.json}
%npm i --package-lock-only and npm ci
%https://docs.npmjs.com/cli/v8/configuring-npm/package-lock-json
%package-lock.json
\end{tabular}
\end{center}

%Dependency Verification

%\begin{tabular}{llll}
%https://docs.gradle.org/current/userguide/dependency_verification.html
%\end{tabular}

Even worse: LaTeX packages (tlmgr and \texttt{DEPENDS.txt} but no notion of pinning),
MATLAB (no widely-used package manager?!).

\begin{block}{Takeway}
Learn how to pin dependencies to \emph{specific versions} or URLs and hashes,
in your context, to make your research easier to
reproduce.
\end{block}

\end{frame}

\begin{frame}{Nix}
\begin{itemize}
  \item{A package manager and build tool.}
  \item{Over 70.000 packages in the main package repository, language-specific repositories (PyPI, Maven Central, \dots) can be wrapped.}
  \item{Used to create the Linux distribution NixOS.}
  \item{Free software, receives funding from NLnet, European Commission.}
  \item{Difficult to use (for now).}
  \item{Storage model departs from the Linux Filesystem Hierarchy Standard.}
\end{itemize}
\end{frame}

\begin{frame}[standout]
Questions?
\end{frame}

\end{document}
